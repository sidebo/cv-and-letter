%----------------------------------------------------------------------------------------
%	SECTION TITLE
%----------------------------------------------------------------------------------------

\cvsection{Arbetslivserfarenhet}

%----------------------------------------------------------------------------------------
%	SECTION CONTENT
%----------------------------------------------------------------------------------------

\begin{cventries}


\cventry
{Doktorand i experimentell partikelfysik} % Degree
{Kungliga Tekniska H{\"o}gskolan} % Institution
{Stockholm och Geneve} % Location
{November 2013 - september 2018} % Date(s)
{ % Description(s) bullet points
\begin{cvitems}
\item {Mitt huvudsakliga forskningsprojekt bestod i att tillsammans med min internationella arbetsgrupp i ATLAS-kollaborationen m{\"a}ta egenskaper hos den nyuppt{\"a}ckta Higgspartikeln. ATLAS-experimentet {\"a}r bel{\"a}get vid den stora hadronkollideraren Large Hadron Collider vid CERN utanf{\"o}r Geneve.
\item Forskningen bestod till stor del av statistisk analys och visualisering av ATLAS stora dataset. Jag ledde under tv{\aa} {\aa}r delprojektet att uppskatta en s{\"a}rskild kontaminering av dataprovet. Denna komponent var helt n{\"o}dv{\"a}ndig f{\"o}r att n{\aa} slutresultatet och kunna publicera. \\ 
$\rightarrow$  \href{http://kth.diva-portal.org/smash/record.jsf?pid=diva2\%3A1244395\&dswid=7018}{{\bf \texttt{[L{\"a}nkAvhandling]}}} \href{http://www.fysikersamfundet.se/wp-content/uploads/Fysikaktuellt3-18_Webb.pdf}{\texttt{[L{\"a}nk-Popul{\"a}rvetenskaplig-Fysikaktuellt (s. 8)]}} }
\item Blev under projektet en central gestalt i gruppen och en expert inom mitt delprojekt, en person andra r{\aa}dfr{\aa}gade och var beroende av.
%\item {Avhandlingens titel: Measurements of the Standard Model Higgs boson cross sections in the $WW$ decay mode with the ATLAS experiment.}
\item {Har bott i Geneve-omr{\aa}det och arbetat p{\aa} plats p{\aa} CERN under sammanlagt ett {\aa}r 2016-2017.} %Fr{\aa}n h{\"o}sten 2016 till v{\aa}ren 2017, samt n{\aa}gra m{\aa}nader under 2015, bodde jag i Geneve-omr{\aa}det och arbetade p{\aa} plats p{\aa} CERN. }
% \item {Under doktorandtiden har jag med gl{\"a}dje {\"a}ven arbetat med undervisning i form av utveckling och ansvar f{\"o}r labbar.}
\item {Erfarenhet av artificiella neurala n{\"a}tverk som anv{\"a}ndes till klassificering av pixelkluster i ATLAS. \\
$\rightarrow$  \href{https://pos.sissa.it/276/213/pdf}{{\bf \texttt{[L{\"a}nkProceeding]}}}}
\item Stor vana att dokumentera arbete och presentera det p{\aa} olika niv{\aa}er. 
		Har presenterat uppdateringar p{\aa} video-baserade veckom{\"o}ten, f{\"o}r st{\"o}rre publik p{\aa} konferenser vid 3-4 tillf{\"a}llen, popul{\"a}rvetenskapliga dragningar vid 3-4 tillf{\"a}llen.
		Jag {\"a}r bra p{\aa} att s{\"a}tta mig in i mottagarens perspektiv och anpassa presentation d{\"a}refter.
\item Har undervisat studenter i fysikkurs om str{\aa}lning,  detektionstekniker och om statistik och dataanalys, totalt 200 timmar i labbet. Jag har varit del av att utvecklingsarbetet av labbar. Ansvar f{\"o}r utvecklingen av en datorbaserad partikelfysik-labb: lade en m{\aa}nad p{\aa} att s{\"a}tta upp den med hj{\"a}lp av Docker och jupyter notebooks. Mycket uppskattat resultat med ett koncept som kommer att {\aa}teranv{\"a}ndas och vidareutvecklas.
\item Under sex m{\aa}nader handledde jag en masterstudent i v{\aa}r grupp, som senare fick betyg A p{\aa} sitt arbete.
\item Har tagit kurser som del av forskarutbildningen, däribland en pedagogikkurs, Advanced Methods in Statistical Data Analysis och CERN School of Computing.
\item Tillsammans med min kollega organiserade vi en mycket uppskattad eftermiddag med {\"o}vningar och f{\"o}rel{\"a}sningar f{\"or} elever p{\aa} min gamla gymnasieskola.
\end{cvitems}
}


%------------------------------------------------

\cventry
{Utredare} % Job title
{Sundsvall Eln\"{a}t} % Organization
{Sundsvall} % Location
{Sommaren 2012, {\aa}rsslutet 2012/2013} % Date(s)
{ % Description(s) of tasks/responsibilities
\begin{cvitems}
\item {P{\aa} eln{\"a}tf{\"o}retaget i min hemstad arbetade jag med ekonomisk dimensionering av kablar och analys av n{\"a}t-regleringsmodell.
		Arbetade sj{\"a}lvst{\"a}ndigt med uppgifter som inte var del av k{\"a}rnverksamheten.
		L{\"a}rde mig att ansvara f{\"o}r mitt eget arbete och att {\"o}vertyga mig sj{\"a}lv och andra om resultat och slutsatser.}
\end{cvitems}
}

%------------------------------------------------

\cventry
{L{\"a}rare} % Job title
{NTI-gymnasiet m.fl.} % Organization
{Ume{\aa}} % Location
{H{\"o}sten 2013} % Date(s)
{ % Description(s) of tasks/responsibilities
\begin{cvitems}
\item {L{\"a}rarvikarie, huvudsakligen p{\aa} gymnasiet i fysik och kemi. L{\"a}rde mig att anpassa mig till olika situationer beroende p{\aa} {\aa}rskurs, {\"a}mne och skola. 
			Efter att timvikarierat med gott resultat fick jag p{\aa} NTI-gymnasiet ett l{\"a}ngre kontrakt p{\aa} ett par m{\aa}nader med ansvar f{\"o}r elev med s{\"a}rskilda behov i kemikunskap. }
\end{cvitems}
}

\cventry
{L{\"a}rare (volont{\"a}r)} % Job title
{Helambu Project} % Organization
{Gangkharka, Nepal} % Location
{Februari - Maj 2011} % Date(s)
{ % Description(s) of tasks/responsibilities
\begin{cvitems}
\item {Undervisning i matte och fysik p{\aa} internatskola i Himalayansk bergsby, f{\"o}r barn i {\aa}ldrarna 7-15 {\aa}r. 
			Jag l{\"a}rde mig att anv{\"a}nda fantasin och anpassa undervisningen f{\"o}r att fungera i en milj{\"o} med knappa resurser.}
\end{cvitems}
}

\end{cventries}