%----------------------------------------------------------------------------------------
%	SECTION TITLE
%----------------------------------------------------------------------------------------

\cvsection{Arbetslivserfarenhet}

%----------------------------------------------------------------------------------------
%	SECTION CONTENT
%----------------------------------------------------------------------------------------

\begin{cventries}


\cventry
{Doktorand i experimentell partikelfysik} % Degree
{Kungliga Tekniska Högskolan} % Institution
{Stockholm och Geneve} % Location
{November 2013 - september 2018} % Date(s)
{ % Description(s) bullet points
\begin{cvitems}
\item {Mitt huvudsakliga forskningsprojekt bestod i att tillsammans med min internationella arbetsgrupp i ATLAS-kollaborationen mäta egenskaper hos den nyupptäckta Higgspartikeln. ATLAS-experimentet är beläget vid den stora hadronkollideraren Large Hadron Collider vid CERN utanför Geneve.
\item Forskningen bestod till stor del av statistisk analys och visualisering av ATLAS stora dataset. Jag ledde under två år delprojektet att uppskatta en särskild kontaminering av dataprovet. Denna komponent var helt nödvändig för att nå slutresultatet och kunna publicera. \\ 
$\rightarrow$  \href{http://kth.diva-portal.org/smash/record.jsf?pid=diva2\%3A1244395\&dswid=7018}{{\bf \texttt{[LänkAvhandling]}}} \href{http://www.fysikersamfundet.se/wp-content/uploads/Fysikaktuellt3-18_Webb.pdf}{\texttt{[Länk-Populärvetenskaplig-Fysikaktuellt (s. 8)]}} }
\item Blev under projektet en central gestalt i gruppen och en expert inom mitt delprojekt, en person andra rådfrågade och var beroende av.
%\item {Avhandlingens titel: Measurements of the Standard Model Higgs boson cross sections in the $WW$ decay mode with the ATLAS experiment.}
\item {Har bott i Geneve-området och arbetat på plats på CERN under sammanlagt ett år 2016-2017.} %Från hösten 2016 till våren 2017, samt några månader under 2015, bodde jag i Geneve-området och arbetade på plats på CERN. }
% \item {Under doktorandtiden har jag med glädje även arbetat med undervisning i form av utveckling och ansvar för labbar.}
\item {Erfarenhet av artificiella neurala nätverk som användes till klassificering av pixelkluster i ATLAS. \\
$\rightarrow$  \href{https://pos.sissa.it/276/213/pdf}{{\bf \texttt{[LänkProceeding]}}}}
\item Stor vana att dokumentera arbete och presentera det på olika nivåer. 
		Har presenterat uppdateringar på video-baserade veckomöten, för större publik på konferenser vid 3-4 tillfällen, populärvetenskapliga dragningar vid 3-4 tillfällen.
		Jag är bra på att sätta mig in i mottagarens perspektiv och anpassa presentation därefter.
\item Har undervisat studenter i fysikkurs om strålning,  detektionstekniker och om statistik och dataanalys, totalt 200 timmar i labbet. Jag har varit del av att utvecklingsarbetet av labbar. Ansvar för utvecklingen av en datorbaserad partikelfysik-labb: lade en månad på att sätta upp den med hjälp av Docker och jupyter notebooks. Mycket uppskattat resultat med ett koncept som kommer att återanvändas och vidareutvecklas.
\item Under sex månader handledde jag en masterstudent i vår grupp, som senare fick betyg A på sitt arbete.
\item Har tagit kurser som del av forskarutbildningen, däribland en pedagogikkurs, Advanced Methods in Statistical Data Analysis och CERN School of Computing (goda programmeringstekniker, IT-säkerhet, datalagring, statistisk dataanalys).
\item Tillsammans med min kollega organiserade vi en mycket uppskattad eftermiddag med övningar och föreläsningar f{\"or} elever på min gamla gymnasieskola.
\end{cvitems}
}


%------------------------------------------------

\cventry
{Utredare} % Job title
{Sundsvall Eln\"{a}t} % Organization
{Sundsvall} % Location
{Sommaren 2012, årsslutet 2012/2013} % Date(s)
{ % Description(s) of tasks/responsibilities
\begin{cvitems}
\item {På elnätföretaget i min hemstad arbetade jag med ekonomisk dimensionering av kablar och analys av nät-regleringsmodell.
		Arbetade självständigt med uppgifter som inte var del av kärnverksamheten.
		Lärde mig att ansvara för mitt eget arbete och att övertyga mig själv och andra om resultat och slutsatser.}
\end{cvitems}
}

%------------------------------------------------

\cventry
{Lärare} % Job title
{NTI-gymnasiet m.fl.} % Organization
{Umeå} % Location
{Hösten 2013} % Date(s)
{ % Description(s) of tasks/responsibilities
\begin{cvitems}
\item {Lärarvikarie, huvudsakligen på gymnasiet i fysik och kemi. Lärde mig att anpassa mig till olika situationer beroende på årskurs, ämne och skola. 
			Efter att timvikarierat med gott resultat fick jag på NTI-gymnasiet ett längre kontrakt på ett par månader med ansvar för elev med särskilda behov i kemikunskap. }
\end{cvitems}
}

\cventry
{Lärare (volontär)} % Job title
{Helambu Project} % Organization
{Gangkharka, Nepal} % Location
{Februari - Maj 2011} % Date(s)
{ % Description(s) of tasks/responsibilities
\begin{cvitems}
\item {Undervisning i matte och fysik på internatskola i Himalayansk bergsby, för barn i åldrarna 7-15 år. 
			Jag lärde mig att använda fantasin och anpassa undervisningen för att fungera i en miljö med knappa resurser.}
\end{cvitems}
}

\end{cventries}