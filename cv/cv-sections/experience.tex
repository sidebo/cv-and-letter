%----------------------------------------------------------------------------------------
%	SECTION TITLE
%----------------------------------------------------------------------------------------

\cvsection{Work experience}

%----------------------------------------------------------------------------------------
%	SECTION CONTENT
%----------------------------------------------------------------------------------------

\begin{cventries}


\cventry
{PhD experimental particle physics} % Degree
{Royal Institute of Technology} % Institution
{Stockholm and Geneva} % Location
{November 2013 - September 2018} % Date(s)
{ % Description(s) bullet points
\begin{cvitems}
\item {Part of the ATLAS experiment at particle physics laboratory CERN outside Geneva.
By analysing data from the proton collisions produced by the Large Hadron Collider me and my group measured properties of the newly discovered Higgs particle. 
The research consisted largely of data analysis, visualisation and application of statistical methods on the large ATLAS dataset. 
For two years I led the project to estimate one of the contaminations of the dataset.
This estimate and its corresponding uncertainty was a key component without which publication was not possible. \\
$\rightarrow$  \href{http://kth.diva-portal.org/smash/record.jsf?pid=diva2\%3A1244395\&dswid=7018}{{\bf \texttt{[Link-Thesis]}}} \href{http://www.fysikersamfundet.se/wp-content/uploads/Fysikaktuellt3-18_Webb.pdf}{\texttt{[Link-PopularScienceArticle-Fysikaktuellt (p. 8)]}}
}
\item {During the time I became an expert within my subfield and was a key person relied on in the group. 
}
\item {Lived in the Geneva area and worked at CERN during about one year 2016-2017.
}
\item {Experience of working with artificial neural networks used for classifying  pixel clusters in the ATLAS detector.
 \\
$\rightarrow$  \href{https://pos.sissa.it/276/213/pdf}{{\bf \texttt{[Link-Proceeding]}}}
}
\item {Much experience of documenting work and presenting it at different levels.
	Often presented at weekly, video-based meetings, more formally for larger audience at conferences on 3-4 occasions, popular science talks at 3-4 occasions.
	I am good at putting myself into the listeners perspective and adapt a presentation accordingly.
}
\item {Have taught undergraduate physics students in laboration exercises about radiation, detection techniques, statistics, data analysis. 
       In total about 200 hours in the lab, in addition to time spent developing and improving the classes.
       I was responsible for developing a computer based particle physics lab: spent one month setting it up with a Docker plus jupyter notebook solution.
       The concept was appreciated by students and teachers who are now taking it further to be used in future courses.
}
\item {During six months I supervised a master student in our group at the university, who later got the highest grade (A) on her work.
}
\item {As part of the PhD I have taken courses, e.g. the CERN School of Computing, and Advanced Methods in Statistical Data Analysis.
}
\item {Organised a ``particle physics afternoon'' (talks, quizzes and hands-on exercises) for students at my old high school together with my colleague.
         The teachers were very pleased with the event which engaged and inspired the students.
}
\end{cvitems}
}


%------------------------------------------------

\cventry
{Investigator} % Job title
{Sundsvall Eln\"{a}t} % Organization
{Sundsvall} % Location
{Summer 2012, year-end 2012/2013} % Date(s)
{ % Description(s) of tasks/responsibilities
\begin{cvitems}
\item {At the electric power grid company in my home town I worked with optimising the size of cables and analysed a network regulation model. 
		 I worked independently with tasks that were not part of the core activity. 
		 I learned to take responsibility for my own work and to convince myself and others about results and conclusions. 
}
\end{cvitems}
}

%------------------------------------------------

\cventry
{Teacher} % Job title
{NTI-gymnasiet and others} % Organization
{Ume{\aa}} % Location
{Fall 2013} % Date(s)
{ % Description(s) of tasks/responsibilities
\begin{cvitems}
\item {Teacher substitute, mostly at upper secondary level in chemistry and physics. 
		I learned to adapt to different situations and students. 
		After being a substitute I got a contract on a few months to help a student with special needs in chemistry.
		 }
\end{cvitems}
}

\cventry
{Teacher (voluntary work)} % Job title
{Helambu Project} % Organization
{Gangkharka, Nepal} % Location
{February - May 2011} % Date(s)
{ % Description(s) of tasks/responsibilities
\begin{cvitems}
\item {Teaching in math and physics on a boarding school in a mountain village in Himalaya, for children in ages 7-15. 
		I learned to use my imagination and set up the curriculum my self, to make it work in an environment with scarce resources.
}
\end{cvitems}
}

\end{cventries}