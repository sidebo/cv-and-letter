%----------------------------------------------------------------------------------------
%	SECTION TITLE
%----------------------------------------------------------------------------------------

\cvsection{Work experience}

%----------------------------------------------------------------------------------------
%	SECTION CONTENT
%----------------------------------------------------------------------------------------

\begin{cventries}


\cventry
{PhD student experimental particle physics} % Degree
{Royal Institute of Technology} % Institution
{Stockholm and Geneva} % Location
{November 2013 - September 2018} % Date(s)
{ % Description(s) bullet points
\begin{cvitems}
\item {Part of the ATLAS experiment at particle physics laboratory CERN outside Geneva.
By analysing data from the proton collisions produced by the Large Hadron Collider me and my group measured properties of the newly discovered Higgs particle. 
The research consisted largely of data analysis, visualisation and application of statistical methods on the large ATLAS dataset. 
For two years I led the project to estimate one of the contaminations of the dataset.
This estimate and its corresponding uncertainty was a key component without which publication was not possible. \\
$\rightarrow$  \href{http://kth.diva-portal.org/smash/record.jsf?pid=diva2\%3A1244395\&dswid=7018}{{\bf \texttt{[Link-Thesis]}}} \href{http://www.fysikersamfundet.se/wp-content/uploads/Fysikaktuellt3-18_Webb.pdf}{\texttt{[Link-PopularScienceArticle-Fysikaktuellt (p. 8)]}}
}
\item {During the time I became an expert within my subfield and was a key person relied on in the group. 
}
\item {Lived in the Geneva area and worked at CERN during about one year 2016-2017.
}
\item {Experience of working with artificial neural networks used for classifying  pixel clusters in the ATLAS detector.
 \\
$\rightarrow$  \href{https://pos.sissa.it/276/213/pdf}{{\bf \texttt{[Link-Proceeding]}}}
}
\item {Much experience of documenting work and presenting it at different levels.
	Often presented at weekly, video-based meetings, more formally for larger audience at conferences on 3-4 occasions, popular science talks at 3-4 occasions.
	I am good at putting myself into the listeners perspective and adapt a presentation accordingly.
}
\item {Have taught undergraduate physics students in laboration exercises about radiation, detection techniques, statistics, data analysis. 
       In total about 200 hours in the lab, in addition to time spent developing and improving the classes.
       I was responsible for developing a computer based particle physics lab: spent one month setting it up with a Docker plus jupyter notebook solution.
       The concept was appreciated by students and teachers who are now taking it further to be used in future courses.
}
\item {During six months I supervised a master student in our group at Royal Institute of Technology, who later got an A on her work.
}
\item {Organised a ``particle physics afternoon'' (talks, quizzes and hands-on exercises) for students at my old high school together with my colleague.
         The teachers were very pleased with the event which engaged and inspired the students a lot.
}
\end{cvitems}
}


%------------------------------------------------

\cventry
{Utredare} % Job title
{Sundsvall Eln\"{a}t} % Organization
{Sundsvall} % Location
{Sommaren 2012, {\aa}rsslutet 2012/2013} % Date(s)
{ % Description(s) of tasks/responsibilities
\begin{cvitems}
\item {P{\aa} eln{\"a}tf{\"o}retaget i min hemstad arbetade jag med ekonomisk dimensionering av kablar och analys av n{\"a}t-regleringsmodell.
		Arbetade sj{\"a}lvst{\"a}ndigt med uppgifter som inte var del av k{\"a}rnverksamheten.
		L{\"a}rde mig att ansvara f{\"o}r mitt eget arbete och att {\"o}vertyga mig sj{\"a}lv och andra om resultat och slutsatser.}
\end{cvitems}
}

%------------------------------------------------

\cventry
{L{\"a}rare} % Job title
{NTI-gymnasiet m.fl.} % Organization
{Ume{\aa}} % Location
{H{\"o}sten 2013} % Date(s)
{ % Description(s) of tasks/responsibilities
\begin{cvitems}
\item {L{\"a}rarvikarie, huvudsakligen p{\aa} gymnasiet i fysik och kemi. L{\"a}rde mig att anpassa mig till olika situationer beroende p{\aa} {\aa}rskurs, {\"a}mne och skola. 
			Efter att timvikarierat med gott resultat fick jag p{\aa} NTI-gymnasiet ett l{\"a}ngre kontrakt p{\aa} ett par m{\aa}nader med ansvar f{\"o}r elev med s{\"a}rskilda behov i kemikunskap. }
\end{cvitems}
}

\cventry
{L{\"a}rare (volont{\"a}r)} % Job title
{Helambu Project} % Organization
{Gangkharka, Nepal} % Location
{Februari - Maj 2011} % Date(s)
{ % Description(s) of tasks/responsibilities
\begin{cvitems}
\item {Undervisning i matte och fysik p{\aa} internatskola i Himalayansk bergsby, f{\"o}r barn i {\aa}ldrarna 7-15 {\aa}r. 
			Jag l{\"a}rde mig att anv{\"a}nda fantasin och anpassa undervisningen f{\"o}r att fungera i en milj{\"o} med knappa resurser.}
\end{cvitems}
}

\end{cventries}