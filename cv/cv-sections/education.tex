%----------------------------------------------------------------------------------------
%	SECTION TITLE
%----------------------------------------------------------------------------------------

\cvsection{Utbildning}

%----------------------------------------------------------------------------------------
%	SECTION CONTENT
%----------------------------------------------------------------------------------------

\begin{cventries}

%------------------------------------------------

%\cventry
%{Doktorandstudier i experimentell partikelfysik} % Degree
%{Kungliga Tekniska Högskolan} % Institution
%{Stockholm och Geneve} % Location
%{November 2013 - september 2018} % Date(s)
%{ % Description(s) bullet points
%\begin{cvitems}
%\item {Mitt huvudsakliga forskningsprojekt bestod i att tillsammans med min arbetsgrupp i ATLAS-kollaborationen mäta hur starkt den nyupptäckta Higgspartikeln, en av naturens fundamentala byggstenar, växelverkar med andra partiklar. ATLAS-experimentet är beläget vid den stora hadronkollideraren Large Hadron Collider vid CERN utanför Geneve. Forskningen består till stor del av dataanalys och visualisering av ATLAS stora dataset. \href{http://kth.diva-portal.org/smash/record.jsf?pid=diva2\%3A1244395\&dswid=7018}{{\bf \texttt{[LänkAvhandling]}}} }
%%\item {Avhandlingens titel: Measurements of the Standard Model Higgs boson cross sections in the $WW$ decay mode with the ATLAS experiment.}
%\item {Från hösten 2016 till våren 2017, samt några månader under 2015, bodde jag i Geneve-området och arbetade på plats på CERN. }
%\item {Under doktorandtiden har jag med glädje även arbetat med undervisning i form av utveckling och ansvar för labbar.}
%\end{cvitems}
%}

%------------------------------------------------

\cventry
{Civilingenjör Teknisk Fysik} % Degree
{Kungliga Tekniska Högskolan} % Institution
{Stockholm} % Location
{2007 - 2013} % Date(s)
{ % Description(s) bullet points
\begin{cvitems}
\item {Masterinriktning: subatomär och astrofysik, med uppsats om Higgspartikeln.}
\item {Ett urval av lästa kurser: Numeriska metoder, Grundläggande datalogi, Programkonstruktion, Kvantfysik, Kärnfysik, Subatomär fysik, Industriell Ekologi, Hållbar Utveckling.}
\end{cvitems}
}

%----------------------------------------------
\cventry
{Fristående kurser} % Degree
{Umeå universitet och Södertörns högskola} % Institution
{Umeå och Stockholm} % Location
{Hösten 2011 och hösten 2012} % Date(s)
{ % Description(s) bullet points
\begin{cvitems}
\item {Av personligt intresse har jag läst en kurs i filosofins historia (Södertörn) och ekonomisk historia (Umeå).}
\end{cvitems}
}

\end{cventries}