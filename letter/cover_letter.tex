%%%%%%%%%%%%%%%%%%%%%%%%%%%%%%%%%%%%%%%%%
% Awesome Cover Letter
% XeLaTeX Template
% Version 1.1 (9/1/2016)
%
% This template has been downloaded from:
% http://www.LaTeXTemplates.com
%
% Original authors:
% Claud D. Park (posquit0.bj@gmail.com)
% Lars Richter (mail@ayeks.de)
% With modifications by:
% Vel (vel@latextemplates.com)
%
% License:
% CC BY-NC-SA 3.0 (http://creativecommons.org/licenses/by-nc-sa/3.0/)
%
% Important note:
% This template must be compiled with XeLaTeX, the below lines will ensure this
%!TEX TS-program = xelatex
%!TEX encoding = UTF-8 Unicode
%
%%%%%%%%%%%%%%%%%%%%%%%%%%%%%%%%%%%%%%%%%

%----------------------------------------------------------------------------------------
%	PACKAGES AND OTHER DOCUMENT CONFIGURATIONS
%----------------------------------------------------------------------------------------

\documentclass[11pt, a4paper]{../awesome-cv} % A4 paper size by default, use 'letterpaper' for US letter

\usepackage[utf8]{inputenc}
\usepackage[swedish]{babel}

\geometry{left=3.5cm, top=1.5cm, right=3.5cm, bottom=2cm, footskip=.5cm} % Configure page margins with geometry
 
\fontdir[fonts/] % Specify the location of the included fonts

% Color for highlights
\colorlet{awesome}{awesome-skyblue} % Default colors include: awesome-emerald, awesome-skyblue, awesome-red, awesome-pink, awesome-orange, awesome-nephritis, awesome-concrete, awesome-darknight
%\definecolor{awesome}{HTML}{CA63A8} % Uncomment if you would like to specify your own color

% Colors for text - uncomment and modify
%\definecolor{darktext}{HTML}{414141}
%\definecolor{text}{HTML}{414141}
%\definecolor{graytext}{HTML}{414141}
%\definecolor{lighttext}{HTML}{414141}

\renewcommand{\acvHeaderSocialSep}{\quad\textbar\quad} % If you would like to change the social information separator from a pipe (|) to something else

%----------------------------------------------------------------------------------------
%	PERSONAL INFORMATION
%	Comment any of the lines below if they are not required
%----------------------------------------------------------------------------------------

\name{Edvin}{Sidebo}
\address{Stockholm}
\mobile{+46 (0)705-485341}

\email{sidebo@kth.se}
%\homepage{www.posquit0.com}
%\github{posquit0}
\linkedin{edvin-sidebo-81abb373}

%\skype{skypeid}
%\stackoverflow{SOid}{SOname}
%\twitter{@twit}
%\linkedin{linkedin name}
%\reddit{reddit account}
%\xing{xing name}
%\extrainfo{test} % Other text you want to include on this line

\position{Doktor experimentell partikelfysik} % Your expertise/fields
%\quote{``Make the change that you want to see in the world."} % A quote or statement

%----------------------------------------------------------------------------------------
%	RECIPIENT/POSITION/LETTER INFORMATION
%	All of the below lines must be filled out
%----------------------------------------------------------------------------------------

\recipient{Kära Ola och Anna,}{} % The company being applied to

\letterdate{\today} % The date on the letter, default is the date of compilation

\lettertitle{} % The title of the letter

\letteropening{} % How the letter is opened

\letterclosing{Bästa h{\"a}lsningar,} % How the letter is closed

\letterenclosure[Bifogat]{CV, akademisk meritlista} % Any enclosures with the letter

%\makecvfooter{\today}{Claud D. Park~~~·~~~Cover Letter}{} % Specify the letter footer with 3 arguments: (<left>, <center>, <right>), leave any of these blank if they are not needed
  
%----------------------------------------------------------------------------------------

\begin{document}

\makecvheader % Print the header

\makelettertitle % Print the title

%----------------------------------------------------------------------------------------
%	LETTER CONTENT
%----------------------------------------------------------------------------------------

\begin{cvletter}

%------------------------------------------------

%\lettersection{About Me}

%Då Hans Rosling och Gapminder började synas i etern för ett antal år sedan kunde jag, i likhet med de flesta, inte annat än bli tagen.  av Hans Roslings fantastiska presentationer och levande  och  sätt
Det var svårt att inte bli tagen av Hans Rosling och hans fantastiska presentationer när de började dyka upp i etern för några år sedan.  
%Likt de flesta blev jag förvånad över hur tillståndet i världen vad gäller grundläggande saker som folkhälsa och utbildning faktiskt ligger
Men kanske ännu mer imponerad blir jag av den otroliga folkbildningsinsats som genomförts sen dess och fortgår tack vare Gapminders arbete. % (inte minst med Factfulness som nu ligger på mitt nattygsbord). 
Detta är ju så viktigt–utan att känna till hur tillståndet i världen ser ut riskerar vi att jobba mot fel mål och med fel metoder. %centrala delar av samhället inte 
%Jag delar er strävan och vision, för mig är dessa en sådan motivator jag behöver på ett arbete.
Jag är civilingenjör i teknisk fysik och har nyligen disputerat i experimentell partikelfysik, men har under senare år blivit alltmer intresserad av samhällsfrågor.
Insikten om att jag i mitt arbete hellre vill ha ett samhälleligt än ett teknologiskt fokus har infunnit sig.
%Jag behöver en stark motivator i mitt arbete–
Jag kontaktar just er för att jag delar er strävan och vision.

Som doktorand var jag medlem i ett av experimenten vid partikelfysiklaboratoriet CERN utanför Geneve.
Arbete gick i huvudsak ut på att filtrera och analysera de enorma datamängder mitt experiment samlat, med hjälp av programmering och användning av statistiska metoder.
Under denna tid insåg jag vad Gapminder gjort till sitt signum: att nå ut med sitt budskap kräver en effektfull presentation. 
%Ett viktigt forskningsresultat i all ära–utan en snygg och levande visualisering är det ändå svårt att tränga igenom.
Jag har därmed fått stor vana av dataanalys och statistisk modellering, samt visualisering och presentation av resultat.
Mycket av arbetet handlade om att självständigt lösa problem med tydliga mål, men utan en tydlig väg.
Jag har därför utvecklat ett resultatorienterat sätt att arbeta.
I forskarutbildningen ingår också undervisningsuppdrag, något som jag med glädje tagit mig an och varit med om att utveckla. 
Min förhoppning är att dessa mina färdigheter kan vara en tillgång i Gapminders arbete.
%Ett exempel på detta är när jag i slutet på min tjänst fick ansvar för att sätta upp en databaserad laborationsuppgift i undervisningssyfte. 
%Efter att ha undersökt olika förslag landade jag i ett Docker-baserat resultat (som eliminerade behovet av programvaruinstallation på studentsidan), som blev mycket uppskattat av min chef och av studenter.

Jag ser att ni inte har några tjänster utlysta för tillfället.
Hoppas att ni trots det vill ta en fortsatt kontakt och diskutera eventuella möjligheter till samarbete!

%------------------------------------------------

\end{cvletter}

%----------------------------------------------------------------------------------------

\makeletterclosing % Print the signature and enclosures

\end{document}