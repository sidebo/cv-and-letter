%%%%%%%%%%%%%%%%%%%%%%%%%%%%%%%%%%%%%%%%%
% Awesome Cover Letter
% XeLaTeX Template
% Version 1.1 (9/1/2016)
%
% This template has been downloaded from:
% http://www.LaTeXTemplates.com
%
% Original authors:
% Claud D. Park (posquit0.bj@gmail.com)
% Lars Richter (mail@ayeks.de)
% With modifications by:
% Vel (vel@latextemplates.com)
%
% License:
% CC BY-NC-SA 3.0 (http://creativecommons.org/licenses/by-nc-sa/3.0/)
%
% Important note:
% This template must be compiled with XeLaTeX, the below lines will ensure this
%!TEX TS-program = xelatex
%!TEX encoding = UTF-8 Unicode
%
%%%%%%%%%%%%%%%%%%%%%%%%%%%%%%%%%%%%%%%%%

%----------------------------------------------------------------------------------------
%	PACKAGES AND OTHER DOCUMENT CONFIGURATIONS
%----------------------------------------------------------------------------------------

\documentclass[11pt, a4paper]{../awesome-cv} % A4 paper size by default, use 'letterpaper' for US letter

\usepackage[utf8]{inputenc}
\usepackage[swedish]{babel}

\geometry{left=3.5cm, top=1.5cm, right=3.5cm, bottom=2cm, footskip=.5cm} % Configure page margins with geometry
 
\fontdir[fonts/] % Specify the location of the included fonts

% Color for highlights
\colorlet{awesome}{awesome-skyblue} % Default colors include: awesome-emerald, awesome-skyblue, awesome-red, awesome-pink, awesome-orange, awesome-nephritis, awesome-concrete, awesome-darknight
%\definecolor{awesome}{HTML}{CA63A8} % Uncomment if you would like to specify your own color

% Colors for text - uncomment and modify
%\definecolor{darktext}{HTML}{414141}
%\definecolor{text}{HTML}{414141}
%\definecolor{graytext}{HTML}{414141}
%\definecolor{lighttext}{HTML}{414141}

\renewcommand{\acvHeaderSocialSep}{\quad\textbar\quad} % If you would like to change the social information separator from a pipe (|) to something else

%----------------------------------------------------------------------------------------
%	PERSONAL INFORMATION
%	Comment any of the lines below if they are not required
%----------------------------------------------------------------------------------------

\name{Edvin}{Sidebo}
\address{Stockholm}
\mobile{+46 (0)705-485341}

\email{sidebo@kth.se}
%\homepage{www.posquit0.com}
%\github{posquit0}
\linkedin{edvin-sidebo-81abb373}

%\skype{skypeid}
%\stackoverflow{SOid}{SOname}
%\twitter{@twit}
%\linkedin{linkedin name}
%\reddit{reddit account}
%\xing{xing name}
%\extrainfo{test} % Other text you want to include on this line

\position{Doktor experimentell partikelfysik} % Your expertise/fields
%\quote{``Make the change that you want to see in the world."} % A quote or statement

%----------------------------------------------------------------------------------------
%	RECIPIENT/POSITION/LETTER INFORMATION
%	All of the below lines must be filled out
%----------------------------------------------------------------------------------------

\recipient{Kära Karin, Birgitta, Pensionsmyndigheten,}{} % The company being applied to

\letterdate{\today} % The date on the letter, default is the date of compilation

\lettertitle{} % The title of the letter

\letteropening{} % How the letter is opened

\letterclosing{Med vänliga hälsningar,} % How the letter is closed

%\letterenclosure[Bifogat]{CV} % Any enclosures with the letter

%\makecvfooter{\today}{Claud D. Park~~~·~~~Cover Letter}{} % Specify the letter footer with 3 arguments: (<left>, <center>, <right>), leave any of these blank if they are not needed
  
%----------------------------------------------------------------------------------------

\begin{document}

\makecvheader % Print the header

\makelettertitle % Print the title

%----------------------------------------------------------------------------------------
%	LETTER CONTENT
%----------------------------------------------------------------------------------------

\begin{cvletter}
\vspace{.6cm}
%------------------------------------------------
% bra på dataanalys, siffror, statistik... analytisk och logiskt tänkande.
% dock bättre än mina fd kollegor på att presentera saker
% samhällsintresse: svenska modellen, medborgares/arbetstagares villkor, vård, utbildning. Flitig lyssnare PoP. 
% har kommit på att det är detta jag vill använda mina kunskaper till. Gillar alltså verkligen vad arenagruppen (och arena idé) gör.
% lite tokig idé: jag skulle kunna vara användbar som kodapa? arena idé skriver många rapporter där man behöver nån som grejar med datat?
% skulle jättegärna komma och träffa dig och diskutera någon form av samarbete.

Jag heter Edvin och är nydisputerad från KTH inom experimentell partikelfysik, där jag varit del av ett stort forskningsprojekt på laboratoriet CERN utanför Geneve.
Nu skulle jag kunna fortsätta min karriär inom forskningen eller söka mig till näringslivet och arbeta med teknologiskt fokus med säg integrerade system eller IT. % eller något annat ingenjörsmässigt.
Men på senare tid har jag blivit allt mer samhällsorienterad och intresserat mig för ekonomi på olika sätt, t.ex. sparande och pensioner. %  inom en idéburen organisation
Det skulle vara motiverande för mig att få en sådan koppling i mitt arbete, hos er skulle jag få det genom att bidra till att göra livet för Sveriges pensionärer enklare.

Jag kommer från forskningen där jag varit del av en stor internationell kollaboration. 
I fem år har jag, tillsammans med andra, ägnat mig åt dataanalys, statistisk modellering, programmering och hantering av enorma dataset.
Under doktorandtiden ledde jag ett delprojekt i en mätning av en av naturens elementarpartiklar.
Ansvaret bestod i att uppskatta en särskild kontaminering av dataprovet, vilken var en helt nödvändig ingrediens för att nå slutresultatet och kunna publicera.
Arbetet gick i huvudsak ut på att skriva dataprogram för att filtrera och visualisera data, och på ett resultatorienterat sätt utveckla och applicera metoder för att utnyttja datans fulla potential.
%Ofta användes avancerade, multivariata, statistiska metoder för att på ett korrekt sätt 
%Ett nyckelord var grundlighet: vi måste vara ärliga med hur mycket
Jag har också fått stor vana av att samarbeta med olika människor och av att presentera på olika nivåer, jag ser som en av mina styrkor förmågan att lägga en presentation på rätt nivå. % och samarbete med olika människor.
 Skrivande är något jag uppskattar och har en fallenhet för. 
%Vidare har jag som forskare också utvecklat en färdighet och uppskattning för skrivande, både av mer akademisk art och på mer tillgängligt språk.

Tyvärr har jag inte erfarenhet av just de verktygen ni önskar, i mitt fall har python och C++ använts.
Men då jag känner mig trygg med min förståelse för både statistikens grundläggande principer och avancerade metoder, ser jag inte något oöverstigligt hinder–jag kan absolut lära mig dessa verktyg.
Om det är något som utmärker min tid som doktorand så är det att jag fått utveckla många nya färdigheter och lära mig nya saker, ofta på kort varsel.
Ett exempel på detta är när jag fick lära mig Docker för att utveckla en laborationsuppgift i undervisningssyfte. 
Med denna lösning blev uppgiften enkel att starta upp via webben, och resultatet blev mycket lyckat och uppskattades av både studenter och min chef.


Jag hoppas att få höra från er snart!

\end{cvletter}

%----------------------------------------------------------------------------------------

\makeletterclosing % Print the signature and enclosures

\end{document}