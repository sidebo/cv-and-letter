%%%%%%%%%%%%%%%%%%%%%%%%%%%%%%%%%%%%%%%%%
% Awesome Cover Letter
% XeLaTeX Template
% Version 1.1 (9/1/2016)
%
% This template has been downloaded from:
% http://www.LaTeXTemplates.com
%
% Original authors:
% Claud D. Park (posquit0.bj@gmail.com)
% Lars Richter (mail@ayeks.de)
% With modifications by:
% Vel (vel@latextemplates.com)
%
% License:
% CC BY-NC-SA 3.0 (http://creativecommons.org/licenses/by-nc-sa/3.0/)
%
% Important note:
% This template must be compiled with XeLaTeX, the below lines will ensure this
%!TEX TS-program = xelatex
%!TEX encoding = UTF-8 Unicode
%
%%%%%%%%%%%%%%%%%%%%%%%%%%%%%%%%%%%%%%%%%

%----------------------------------------------------------------------------------------
%	PACKAGES AND OTHER DOCUMENT CONFIGURATIONS
%----------------------------------------------------------------------------------------

\documentclass[11pt, a4paper]{../awesome-cv} % A4 paper size by default, use 'letterpaper' for US letter

\usepackage[utf8]{inputenc}
\usepackage[swedish]{babel}

\geometry{left=3.5cm, top=1.5cm, right=3.5cm, bottom=2cm, footskip=.5cm} % Configure page margins with geometry
 
\fontdir[fonts/] % Specify the location of the included fonts

% Color for highlights
\colorlet{awesome}{awesome-skyblue} % Default colors include: awesome-emerald, awesome-skyblue, awesome-red, awesome-pink, awesome-orange, awesome-nephritis, awesome-concrete, awesome-darknight
%\definecolor{awesome}{HTML}{CA63A8} % Uncomment if you would like to specify your own color

% Colors for text - uncomment and modify
%\definecolor{darktext}{HTML}{414141}
%\definecolor{text}{HTML}{414141}
%\definecolor{graytext}{HTML}{414141}
%\definecolor{lighttext}{HTML}{414141}

\renewcommand{\acvHeaderSocialSep}{\quad\textbar\quad} % If you would like to change the social information separator from a pipe (|) to something else

%----------------------------------------------------------------------------------------
%	PERSONAL INFORMATION
%	Comment any of the lines below if they are not required
%----------------------------------------------------------------------------------------

\name{Edvin}{Sidebo}
\address{Stockholm}
\mobile{+46 (0)705-485341}

\email{sidebo@kth.se}
%\homepage{www.posquit0.com}
%\github{posquit0}
\linkedin{edvin-sidebo-81abb373}

%\skype{skypeid}
%\stackoverflow{SOid}{SOname}
%\twitter{@twit}
%\linkedin{linkedin name}
%\reddit{reddit account}
%\xing{xing name}
%\extrainfo{test} % Other text you want to include on this line

\position{Doktor experimentell partikelfysik} % Your expertise/fields
%\quote{``Make the change that you want to see in the world."} % A quote or statement

%----------------------------------------------------------------------------------------
%	RECIPIENT/POSITION/LETTER INFORMATION
%	All of the below lines must be filled out
%----------------------------------------------------------------------------------------

\recipient{Utredare radioaktivt avfall}{} % The company being applied to

\letterdate{\today} % The date on the letter, default is the date of compilation

%\lettertitle{------------------------------------------------------------------------------------------------------------------------------------------------} % The title of the letter

\letteropening{\vspace{1cm}Kära Eva, Strålsäkerhetsmyndigheten,} % How the letter is opened

\letterclosing{Bästa hälsningar, \\ Edvin} % How the letter is closed

%\letterenclosure[Bifogat]{CV} % Any enclosures with the letter

%\makecvfooter{\today}{Claud D. Park~~~·~~~Cover Letter}{} % Specify the letter footer with 3 arguments: (<left>, <center>, <right>), leave any of these blank if they are not needed
  
%----------------------------------------------------------------------------------------

\begin{document}
\sloppy % use \sloppy to make sure words don't extend into margin

\makecvheader % Print the header

\makelettertitle % Print the title

%----------------------------------------------------------------------------------------
%	LETTER CONTENT
%----------------------------------------------------------------------------------------

\begin{cvletter}
\vspace{.2cm}
%------------------------------------------------
VARFÖR SSM?

Efter att nyligen disputerat från KTH inom experimentell partikelfysik letar jag nu nya utmaningar utanför akademin. %och tror att Elekta kan passa mig väl. 
I min forskning har jag lärt mig mycket som jag är säker på är värdefullt för denna tjänst.
Först och främst är förståelse för strålning och växelverkan med material en central komponent i forskningen.
Den underliggande fysiken bestämmer design och val av detektorer som används i experimenten.
% Att förstå joniserande strålning och dess växelverkan med material är centralt för design och funktion av de detektorer vi bygger och använder inom partikelfysiken. 
Jag har tränats i dessa kunskaper och även lärt ut grunderna till fysikstudenter på KTH i laborationsbaserad undervisning.
Medan många inom forskningen helst vill slippa undervisning har jag istället uppskattat den, jag tycker om att hjälpa och tar mig gärna an sådana utmaningar. % brinner för bra metoder inom inlärning. 

Under doktorandtiden var jag medlem i ATLAS-experimentet, beläget vid partikelfysiklaboratoriet CERN utanför Geneve.
Vidare har jag stor datavana med erfarenhet av programmering och statistisk dataanalys i t.ex. C++ och python. 
%Nedan följer en liten presentation av mig och varför jag tror jag kan vara en bra kandidat.
%Nu letar jag nya utmaningar utanför akademin och tror att Elekta kan passa mig väl. 
Tillsammans med en internationell forskargrupp gjorde vi en mätning av den nyupptäckta Higgspartikeln. 
Jag ledde arbetet med uppskattningen av en särskild kontaminering av dataprovet.
Huvudsakligen bestod jobbet i att analysera de stora datamängder experimentet insamlat, %att övertyga sig själv och andra om att man förstår sin data.
vilket ofta krävde utforskande av olika lösningar på problem och noggrann utvärdering. 
Jag har fått stor erfarenhet av Monte Carlo-simuleringar, t.ex. genom arbetet i ett uppgraderingsprojekt där jag studerade en framtida kiselbaserad tidsdetektor.
%Personligen uppskattade jag programmeringen och lade gärna lite extra tid på att göra kod elegantare och stabilare istället för att hasta till resultat.
Vidare har jag fått stor erfarenhet av att presentera resultat, på olika nivåer, och användning av avancerade statistiska metoder för tolkning av data.




Jag hoppas på att få höra från er snart!

\end{cvletter}

%----------------------------------------------------------------------------------------

\makeletterclosing % Print the signature and enclosures

\end{document}