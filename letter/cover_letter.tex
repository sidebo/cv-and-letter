%%%%%%%%%%%%%%%%%%%%%%%%%%%%%%%%%%%%%%%%%
% Awesome Cover Letter
% XeLaTeX Template
% Version 1.1 (9/1/2016)
%
% This template has been downloaded from:
% http://www.LaTeXTemplates.com
%
% Original authors:
% Claud D. Park (posquit0.bj@gmail.com)
% Lars Richter (mail@ayeks.de)
% With modifications by:
% Vel (vel@latextemplates.com)
%
% License:
% CC BY-NC-SA 3.0 (http://creativecommons.org/licenses/by-nc-sa/3.0/)
%
% Important note:
% This template must be compiled with XeLaTeX, the below lines will ensure this
%!TEX TS-program = xelatex
%!TEX encoding = UTF-8 Unicode
%
%%%%%%%%%%%%%%%%%%%%%%%%%%%%%%%%%%%%%%%%%

%----------------------------------------------------------------------------------------
%	PACKAGES AND OTHER DOCUMENT CONFIGURATIONS
%----------------------------------------------------------------------------------------

\documentclass[11pt, a4paper]{../awesome-cv} % A4 paper size by default, use 'letterpaper' for US letter

\usepackage[utf8]{inputenc}
\usepackage[swedish]{babel}

\geometry{left=3.5cm, top=1.5cm, right=3.5cm, bottom=2cm, footskip=.5cm} % Configure page margins with geometry
 
\fontdir[fonts/] % Specify the location of the included fonts

% Color for highlights
\colorlet{awesome}{awesome-skyblue} % Default colors include: awesome-emerald, awesome-skyblue, awesome-red, awesome-pink, awesome-orange, awesome-nephritis, awesome-concrete, awesome-darknight
%\definecolor{awesome}{HTML}{CA63A8} % Uncomment if you would like to specify your own color

% Colors for text - uncomment and modify
%\definecolor{darktext}{HTML}{414141}
%\definecolor{text}{HTML}{414141}
%\definecolor{graytext}{HTML}{414141}
%\definecolor{lighttext}{HTML}{414141}

\renewcommand{\acvHeaderSocialSep}{\quad\textbar\quad} % If you would like to change the social information separator from a pipe (|) to something else

%----------------------------------------------------------------------------------------
%	PERSONAL INFORMATION
%	Comment any of the lines below if they are not required
%----------------------------------------------------------------------------------------

\name{Edvin}{Sidebo}
\address{Stockholm}
\mobile{+46 (0)705-485341}

\email{sidebo@kth.se}
%\homepage{www.posquit0.com}
%\github{posquit0}
\linkedin{edvin-sidebo-81abb373}

%\skype{skypeid}
%\stackoverflow{SOid}{SOname}
%\twitter{@twit}
%\linkedin{linkedin name}
%\reddit{reddit account}
%\xing{xing name}
%\extrainfo{test} % Other text you want to include on this line

\position{Doktor experimentell partikelfysik} % Your expertise/fields
%\quote{``Make the change that you want to see in the world."} % A quote or statement

%----------------------------------------------------------------------------------------
%	RECIPIENT/POSITION/LETTER INFORMATION
%	All of the below lines must be filled out
%----------------------------------------------------------------------------------------

\recipient{Physicist/Software Developer}{} % The company being applied to

\letterdate{\today} % The date on the letter, default is the date of compilation

%\lettertitle{------------------------------------------------------------------------------------------------------------------------------------------------} % The title of the letter

\letteropening{\vspace{1cm}Kära Erik, RaySearch,} % How the letter is opened

\letterclosing{Bästa hälsningar, \\ Edvin} % How the letter is closed

%\letterenclosure[Bifogat]{CV} % Any enclosures with the letter

%\makecvfooter{\today}{Claud D. Park~~~·~~~Cover Letter}{} % Specify the letter footer with 3 arguments: (<left>, <center>, <right>), leave any of these blank if they are not needed
  
%----------------------------------------------------------------------------------------

\begin{document}
\sloppy % use \sloppy to make sure words don't extend into margin

\makecvheader % Print the header

\makelettertitle % Print the title

%----------------------------------------------------------------------------------------
%	LETTER CONTENT
%----------------------------------------------------------------------------------------

\begin{cvletter}
\vspace{.2cm}
%------------------------------------------------
Låt mig börja med att gratulera till nya RayStation 8B–superspännande att vara pionjärer för maskin-inlärning inom strålbehandling!

I höstas disputerade jag från KTH inom experimentell partikelfysik, där jag var medlem i ATLAS-experimentet, beläget vid partikelfysiklaboratoriet CERN utanför Geneve.
Nu letar jag nya utmaningar utanför akademin och tror att RaySearch kan passa mig väl. 
Under doktorandtiden arbetade jag huvudsakligen i en internationell forskargrupp med en mätning av den nyupptäckta Higgspartikeln. 
Jag ledde arbetet med uppskattningen av en särskild kontaminering av dataprovet.
Rent konkret bestod arbetet till stor del av att skriva program, huvudsakligen i C++, i syfte att analysera de stora datamängder experimentet insamlat.
Personligen uppskattade jag programmeringen och lade gärna lite extra tid på att göra kod elegantare och stabilare istället för att hasta till resultat.
Vidare har jag fått stor erfarenhet av att presentera resultat, på olika nivåer, och användning av statistiska metoder för tolkning av data.

Att förstå joniserande strålning och dess växelverkan med material är centralt för design och funktion av de detektorer vi bygger och använder inom partikelfysiken. 
Jag har tränats i dessa kunskaper och även lärt ut grunderna (tillsammans med Victor Mikhalev som nu jobbar hos er) till fysikstudenter på KTH.

Att få arbeta på RaySearch skulle vara väldigt spännande och motiverande för mig.
Då får jag kombinera mitt intresse för programmering och fysik samtidigt som arbetet bidrar till att bekämpa en av de vanligaste dödsorsakerna.

Jag hoppas på att få höra från er snart!

\end{cvletter}

%----------------------------------------------------------------------------------------

\makeletterclosing % Print the signature and enclosures

\end{document}