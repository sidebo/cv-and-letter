%%%%%%%%%%%%%%%%%%%%%%%%%%%%%%%%%%%%%%%%%
% Awesome Cover Letter
% XeLaTeX Template
% Version 1.1 (9/1/2016)
%
% This template has been downloaded from:
% http://www.LaTeXTemplates.com
%
% Original authors:
% Claud D. Park (posquit0.bj@gmail.com)
% Lars Richter (mail@ayeks.de)
% With modifications by:
% Vel (vel@latextemplates.com)
%
% License:
% CC BY-NC-SA 3.0 (http://creativecommons.org/licenses/by-nc-sa/3.0/)
%
% Important note:
% This template must be compiled with XeLaTeX, the below lines will ensure this
%!TEX TS-program = xelatex
%!TEX encoding = UTF-8 Unicode
%
%%%%%%%%%%%%%%%%%%%%%%%%%%%%%%%%%%%%%%%%%

%----------------------------------------------------------------------------------------
%	PACKAGES AND OTHER DOCUMENT CONFIGURATIONS
%----------------------------------------------------------------------------------------

\documentclass[11pt, a4paper]{../awesome-cv} % A4 paper size by default, use 'letterpaper' for US letter

\usepackage[utf8]{inputenc}
\usepackage[swedish]{babel}

\geometry{left=3.5cm, top=1.5cm, right=3.5cm, bottom=2cm, footskip=.5cm} % Configure page margins with geometry
 
\fontdir[fonts/] % Specify the location of the included fonts

% Color for highlights
\colorlet{awesome}{awesome-skyblue} % Default colors include: awesome-emerald, awesome-skyblue, awesome-red, awesome-pink, awesome-orange, awesome-nephritis, awesome-concrete, awesome-darknight
%\definecolor{awesome}{HTML}{CA63A8} % Uncomment if you would like to specify your own color

% Colors for text - uncomment and modify
%\definecolor{darktext}{HTML}{414141}
%\definecolor{text}{HTML}{414141}
%\definecolor{graytext}{HTML}{414141}
%\definecolor{lighttext}{HTML}{414141}

\renewcommand{\acvHeaderSocialSep}{\quad\textbar\quad} % If you would like to change the social information separator from a pipe (|) to something else

%----------------------------------------------------------------------------------------
%	PERSONAL INFORMATION
%	Comment any of the lines below if they are not required
%----------------------------------------------------------------------------------------

\name{Edvin}{Sidebo}
\address{Stockholm}
\mobile{+46 (0)705-485341}

\email{sidebo@kth.se}
%\homepage{www.posquit0.com}
%\github{posquit0}
\linkedin{edvin-sidebo-81abb373}

%\skype{skypeid}
%\stackoverflow{SOid}{SOname}
%\twitter{@twit}
%\linkedin{linkedin name}
%\reddit{reddit account}
%\xing{xing name}
%\extrainfo{test} % Other text you want to include on this line

\position{Doktor experimentell partikelfysik} % Your expertise/fields
%\quote{``Make the change that you want to see in the world."} % A quote or statement

%----------------------------------------------------------------------------------------
%	RECIPIENT/POSITION/LETTER INFORMATION
%	All of the below lines must be filled out
%----------------------------------------------------------------------------------------

\recipient{Physicist/Software Developer}{} % The company being applied to

\letterdate{\today} % The date on the letter, default is the date of compilation

%\lettertitle{------------------------------------------------------------------------------------------------------------------------------------------------} % The title of the letter

\letteropening{Dear Combient,} % How the letter is opened

\letterclosing{Best regards, \\ Edvin} % How the letter is closed

%\letterenclosure[Bifogat]{CV} % Any enclosures with the letter

%\makecvfooter{\today}{Claud D. Park~~~·~~~Cover Letter}{} % Specify the letter footer with 3 arguments: (<left>, <center>, <right>), leave any of these blank if they are not needed
  
%----------------------------------------------------------------------------------------

\begin{document}
\sloppy % use \sloppy to make sure words don't extend into margin

\makecvheader % Print the header

\makelettertitle % Print the title

%----------------------------------------------------------------------------------------
%	LETTER CONTENT
%----------------------------------------------------------------------------------------

\begin{cvletter}
\vspace{.2cm}
%------------------------------------------------

Nyckelord: 

kollaboration (erfarenhet) - dela med sig, lära sig av varandra

dataanalys och avancerade statistiska modeller -- statistiskt tänkande

programmering och versionshantering

digitalisering 

problemlösning: det finns sällan given väg/lösning. 

------------------------------------------------------------------------

% letter starts here, copy paste below and paste into form
Dear Combient,

I was very inspired by Rahman Amandius when he came to visit my workplace to talk about Combient and how to transition from academia to industrial data science.
Combient seems to me like a place in which one can thrive personally while at the same time providing disrupting solutions with great value for customers.
I also liked to hear about the collaborative spirit which appear to be a key component in the Combient ecosystem.

I recently graduated from the Royal Institute of Technology (KTH) in Stockholm where I did research as part of a large international collaboration based at the particle physics laboratory CERN outside Geneva.
%Part of a group of about 25 people we carried out a measurement of the newly discovered Higgs particle, by producing such particles in the proton collisions made by the Large Hadron Collider.
My main project concerned a measurement of the Higgs particle and largely circled around analysing the large dataset of proton collisions, and solving problems arising in the process of understanding the data.
%The research is largely based on analysing ATLAS' large dataset and solving problems arising in the process of understanding the data.
In this work I have a developed a strong analytical skill set and a creative mind. 
For two years I led the effort to estimate one particular source of contamination of the data, the estimation of which was crucial to reach publication. 
Advanced statistical methods are used to extract the most out of the data.
However, I am mostly glad for the {\it statistical thinking} or attitude I feel I have developed--in my opinion, understanding data relies on having the right mindset, more so than being able to apply a series of methods from a recipe.
This way of thinking is useful in all situations in which one deals with data.

I have strong skills in Python and C++, especially the latter which was the primary language used during my PhD.
I am always eager to learn new things though, which has often been the case in an environment where it is not obvious which solution is the best.
For example, I learned Docker when given the job to develop a laboration exercise for teaching purposes. 
I realised such a solution would be a neat way to eliminate the need for software installation on the student side.

I hope that these skills and my mindset is what you are looking for at Combient, and that I can come to an interview to present myself and learn more about you and the position.

Best regards,
Edvin


\end{cvletter}

%----------------------------------------------------------------------------------------

\makeletterclosing % Print the signature and enclosures

\end{document}