%%%%%%%%%%%%%%%%%%%%%%%%%%%%%%%%%%%%%%%%%
% Awesome Cover Letter
% XeLaTeX Template
% Version 1.1 (9/1/2016)
%
% This template has been downloaded from:
% http://www.LaTeXTemplates.com
%
% Original authors:
% Claud D. Park (posquit0.bj@gmail.com)
% Lars Richter (mail@ayeks.de)
% With modifications by:
% Vel (vel@latextemplates.com)
%
% License:
% CC BY-NC-SA 3.0 (http://creativecommons.org/licenses/by-nc-sa/3.0/)
%
% Important note:
% This template must be compiled with XeLaTeX, the below lines will ensure this
%!TEX TS-program = xelatex
%!TEX encoding = UTF-8 Unicode
%
%%%%%%%%%%%%%%%%%%%%%%%%%%%%%%%%%%%%%%%%%

%----------------------------------------------------------------------------------------
%	PACKAGES AND OTHER DOCUMENT CONFIGURATIONS
%----------------------------------------------------------------------------------------

\documentclass[11pt, a4paper]{../awesome-cv} % A4 paper size by default, use 'letterpaper' for US letter

\usepackage[utf8]{inputenc}
\usepackage[swedish]{babel}

\geometry{left=3.5cm, top=1.5cm, right=3.5cm, bottom=2cm, footskip=.5cm} % Configure page margins with geometry
 
\fontdir[fonts/] % Specify the location of the included fonts

% Color for highlights
\colorlet{awesome}{awesome-skyblue} % Default colors include: awesome-emerald, awesome-skyblue, awesome-red, awesome-pink, awesome-orange, awesome-nephritis, awesome-concrete, awesome-darknight
%\definecolor{awesome}{HTML}{CA63A8} % Uncomment if you would like to specify your own color

% Colors for text - uncomment and modify
%\definecolor{darktext}{HTML}{414141}
%\definecolor{text}{HTML}{414141}
%\definecolor{graytext}{HTML}{414141}
%\definecolor{lighttext}{HTML}{414141}

\renewcommand{\acvHeaderSocialSep}{\quad\textbar\quad} % If you would like to change the social information separator from a pipe (|) to something else

%----------------------------------------------------------------------------------------
%	PERSONAL INFORMATION
%	Comment any of the lines below if they are not required
%----------------------------------------------------------------------------------------

\name{Edvin}{Sidebo}
\address{Stockholm}
\mobile{+46 (0)705-485341}

\email{sidebo@kth.se}
%\homepage{www.posquit0.com}
%\github{posquit0}
\linkedin{edvin-sidebo-81abb373}

%\skype{skypeid}
%\stackoverflow{SOid}{SOname}
%\twitter{@twit}
%\linkedin{linkedin name}
%\reddit{reddit account}
%\xing{xing name}
%\extrainfo{test} % Other text you want to include on this line

\position{Doktor experimentell partikelfysik} % Your expertise/fields
%\quote{``Make the change that you want to see in the world."} % A quote or statement

%----------------------------------------------------------------------------------------
%	RECIPIENT/POSITION/LETTER INFORMATION
%	All of the below lines must be filled out
%----------------------------------------------------------------------------------------

\recipient{Kära Johanna,}{} % The company being applied to

\letterdate{\today} % The date on the letter, default is the date of compilation

\lettertitle{} % The title of the letter

\letteropening{} % How the letter is opened

\letterclosing{God jul och gott nytt år,} % How the letter is closed

\letterenclosure[Bifogat]{CV} % Any enclosures with the letter

%\makecvfooter{\today}{Claud D. Park~~~·~~~Cover Letter}{} % Specify the letter footer with 3 arguments: (<left>, <center>, <right>), leave any of these blank if they are not needed
  
%----------------------------------------------------------------------------------------

\begin{document}

\makecvheader % Print the header

\makelettertitle % Print the title

%----------------------------------------------------------------------------------------
%	LETTER CONTENT
%----------------------------------------------------------------------------------------

\begin{cvletter}
\vspace{.6cm}
%------------------------------------------------
% bra på dataanalys, siffror, statistik... analytisk och logiskt tänkande.
% dock bättre än mina fd kollegor på att presentera saker
% samhällsintresse: svenska modellen, medborgares/arbetstagares villkor, vård, utbildning. Flitig lyssnare PoP. 
% har kommit på att det är detta jag vill använda mina kunskaper till. Gillar alltså verkligen vad arenagruppen (och arena idé) gör.
% lite tokig idé: jag skulle kunna vara användbar som kodapa? arena idé skriver många rapporter där man behöver nån som grejar med datat?
% skulle jättegärna komma och träffa dig och diskutera någon form av samarbete.

Jag heter Edvin och är nydisputerad från KTH inom experimentell partikelfysik.
Nu skulle jag kunna fortsätta min karriär inom forskningen eller söka mig till näringslivet och arbeta med integrerade system, IT eller något annat ingenjörigt.
Men tanken på det skaver, jag känner att det är något annat jag vill använda mina färdigheter till. 
Jag har blivit mer och mer intresserad av samhällsfrågor–det skulle vara enormt motiverande för mig att istället få verka inom denna ideella sfären.

Arenagruppen och kanske främst Arena Idé är en stor anledning till mitt skiftade fokus. %växande glöd. 
Jag har med stort intresse lyssnat på alla avsnitt av Pengar och Politik, vilket fört mig vidare till rapporter och seminarier, t.ex. om marknadiseringen av sjukvården. % Arena Idés rapporter (t.ex. och tittat på seminarier
Min tanke är nu följande: kanske kan ni ha användning för en sådan som mig?
Jag har stor erfarenhet av dataanalys, statistisk modellering och programmering, då jag arbetat med stora dataset.
Som medlem i en internationell organisation under min doktorandtid har jag också stor vana av att presentera på olika nivåer och samarbete med olika människor. 
Vidare har jag som forskare också utvecklat en färdighet och uppskattning för skrivande, både av mer akademisk art och på mer tillgängligt språk.

Jag skulle jättegärna träffa dig och diskutera ett eventuellt samarbete!
%------------------------------------------------

%\lettersection{Why Initech?}
%
%Suspendisse commodo, massa eu congue tincidunt, elit mauris pellentesque orci, cursus tempor odio nisl euismod augue. Aliquam adipiscing nibh ut odio sodales et pulvinar tortor laoreet. Mauris a accumsan ligula. Class aptent taciti sociosqu ad litora torquent per conubia nostra, per inceptos himenaeos. Suspendisse vulputate sem vehicula ipsum varius nec tempus dui dapibus. Phasellus et est urna, ut auctor erat. Sed tincidunt odio id odio aliquam mattis. Donec sapien nulla, feugiat eget adipiscing sit amet, lacinia ut dolor. Phasellus tincidunt, leo a fringilla consectetur, felis diam aliquam urna, vitae aliquet lectus orci nec velit. Vivamus dapibus varius blandit.
%
%%------------------------------------------------
%
%\lettersection{Why Me?}
%
%Duis sit amet magna ante, at sodales diam. Aenean consectetur porta risus et sagittis. Ut interdum, enim varius pellentesque tincidunt, magna libero sodales tortor, ut fermentum nunc metus a ante. Vivamus odio leo, tincidunt eu luctus ut, sollicitudin sit amet metus. Nunc sed orci lectus. Ut sodales magna sed velit volutpat sit amet pulvinar diam venenatis.

%------------------------------------------------

\end{cvletter}

%----------------------------------------------------------------------------------------

\makeletterclosing % Print the signature and enclosures

\end{document}