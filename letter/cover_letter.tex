%%%%%%%%%%%%%%%%%%%%%%%%%%%%%%%%%%%%%%%%%
% Awesome Cover Letter
% XeLaTeX Template
% Version 1.1 (9/1/2016)
%
% This template has been downloaded from:
% http://www.LaTeXTemplates.com
%
% Original authors:
% Claud D. Park (posquit0.bj@gmail.com)
% Lars Richter (mail@ayeks.de)
% With modifications by:
% Vel (vel@latextemplates.com)
%
% License:
% CC BY-NC-SA 3.0 (http://creativecommons.org/licenses/by-nc-sa/3.0/)
%
% Important note:
% This template must be compiled with XeLaTeX, the below lines will ensure this
%!TEX TS-program = xelatex
%!TEX encoding = UTF-8 Unicode
%
%%%%%%%%%%%%%%%%%%%%%%%%%%%%%%%%%%%%%%%%%

%----------------------------------------------------------------------------------------
%	PACKAGES AND OTHER DOCUMENT CONFIGURATIONS
%----------------------------------------------------------------------------------------

\documentclass[11pt, a4paper]{../awesome-cv} % A4 paper size by default, use 'letterpaper' for US letter

\usepackage[utf8]{inputenc}
\usepackage[swedish]{babel}

\hyphenation{data-analys}

\geometry{left=3.5cm, top=1.5cm, right=3.5cm, bottom=2cm, footskip=.5cm} % Configure page margins with geometry
 
\fontdir[fonts/] % Specify the location of the included fonts

% Color for highlights
\colorlet{awesome}{awesome-skyblue} % Default colors include: awesome-emerald, awesome-skyblue, awesome-red, awesome-pink, awesome-orange, awesome-nephritis, awesome-concrete, awesome-darknight
%\definecolor{awesome}{HTML}{CA63A8} % Uncomment if you would like to specify your own color

% Colors for text - uncomment and modify
%\definecolor{darktext}{HTML}{414141}
%\definecolor{text}{HTML}{414141}
%\definecolor{graytext}{HTML}{414141}
%\definecolor{lighttext}{HTML}{414141}

\renewcommand{\acvHeaderSocialSep}{\quad\textbar\quad} % If you would like to change the social information separator from a pipe (|) to something else

%----------------------------------------------------------------------------------------
%	PERSONAL INFORMATION
%	Comment any of the lines below if they are not required
%----------------------------------------------------------------------------------------

\name{Edvin}{Sidebo}
\address{Stockholm}
\mobile{+46 (0)705-485341}

\email{sidebo@kth.se}
%\homepage{www.posquit0.com}
%\github{posquit0}
\linkedin{edvin-sidebo-81abb373}

%\skype{skypeid}
%\stackoverflow{SOid}{SOname}
%\twitter{@twit}
%\linkedin{linkedin name}
%\reddit{reddit account}
%\xing{xing name}
%\extrainfo{test} % Other text you want to include on this line

\position{Doktor experimentell partikelfysik} % Your expertise/fields
%\quote{``Make the change that you want to see in the world."} % A quote or statement

%----------------------------------------------------------------------------------------
%	RECIPIENT/POSITION/LETTER INFORMATION
%	All of the below lines must be filled out
%----------------------------------------------------------------------------------------

\recipient{Ingenjör – Sigma}{} % The company being applied to

\letterdate{\today} % The date on the letter, default is the date of compilation

%\lettertitle{------------------------------------------------------------------------------------------------------------------------------------------------} % The title of the letter

\letteropening{\vspace{1cm}Kära Mathias, Sigma,} % How the letter is opened

\letterclosing{Bästa hälsningar, \\ Edvin} % How the letter is closed

%\letterenclosure[Bifogat]{CV} % Any enclosures with the letter

%\makecvfooter{\today}{Claud D. Park~~~·~~~Cover Letter}{} % Specify the letter footer with 3 arguments: (<left>, <center>, <right>), leave any of these blank if they are not needed
  
%----------------------------------------------------------------------------------------

\begin{document}
\sloppy % use \sloppy to make sure words don't extend into margin

\makecvheader % Print the header

\makelettertitle % Print the title

%----------------------------------------------------------------------------------------
%	LETTER CONTENT
%----------------------------------------------------------------------------------------

\begin{cvletter}
\vspace{.2cm}
%------------------------------------------------
%VARFÖR SSM?

Efter att nyligen disputerat från KTH inom experimentell partikelfysik letar jag nu nya utmaningar utanför akademin. %och tror att Elekta kan passa mig väl. 
Det var exalterande att hitta denna tjänst annonserad hos er.
Jag har hört mycket gott om Sigma (inte minst femteplatsen i Sveriges Bästa Arbetsgivare) och ert fokus på mjukvara skulle passa mig perfekt.
Samtidigt hoppas och tror jag att mina kompetenser skulle gynna er. 

%, det var med glädje jag upptäckte annonsen till denna tjänst. 
%Vidare verkar uppdraget och arbetsuppgifterna passa mig väl
%Det finns gott om exempel i den moderna historien som visar på vikten av strålskyddsarbete och 

I min forskning var jag medlem i ett av experimenten vid partikelfysiklaboratoriet CERN utanför Geneve.
Jag jobbade huvudsakligen med att tillsammans med min internationella forskargrupp göra en mätning av en av naturens fundamentala partiklar (Higgsbosonen), genom att analysera de stora datamängder experimentet samlat in. %gick en av naturens gjorde vi en mätning av den nyupptäckta Higgspartikeln. 
Detta arbete har gett mig mycket som jag tror är värdefullt för denna tjänst.
Genom dataanalysen har jag fått gedigen erfarenhet av programmering i t.ex. C++ och python, liksom statistisk modellering och hantering av stora mängder data.

I projektet ledde jag arbetet med uppskattningen av en särskild kontaminering av dataprovet. % och huvudsakligen bestod jobbet i att analysera de stora datamängder experimentet insamlat. %att övertyga sig själv och andra om att man förstår sin data.
Detta var ett särskilt komplext uppdrag då den vanliga metoden att använda Monte Carlo-simuleringar inte kunde användas som brukligt–istället krävdes utforskande av okonventionella, datadrivna metoder med noggrann utvärdering av resultat. %. utforskande av olika lösningar på problem och noggrann utvärdering av resultat.
%Jag har också fått stor erfarenhet av Monte Carlo-simuleringar, t.ex. genom arbetet i ett uppgraderingsprojekt där jag studerade en framtida kiselbaserad tidsdetektor.
%Personligen uppskattade jag programmeringen och lade gärna lite extra tid på att göra kod elegantare och stabilare istället för att hasta till resultat.
I samarbetet med människor från olika platser och i olika positioner har jag tränats i att presentera resultat, på olika nivåer, från arbetsmöten på veckobasis till populärvetenskapliga dragningar och konferensbidrag.

Mina f.d. kollegor skulle beskriva mig som en stresstålig person som håller humöret uppe, hjälpsam och kapabel, med ett driv för att förstå. 
Jag är inte heller rädd för att vara i en situation där jag behöver lära mig något nytt–tvärtom är det en rolig utmaning. 
Ett exempel på detta är när jag fick uppdraget att utveckla och sätta upp en datorbaserad laborationsuppgift för undervisning.
Efter att ha undersökt olika alternativ bedömde jag att en lösning baserad på Docker, som jag då lärde mig, var ett bra sätt att eliminera behovet av mjukvaruinstallation för studenterna.
Resultatet blev mycket uppskattat av studenter och av min chef och konceptet återanvänds och vidareutvecklas.

Sammantaget är jag övertygad om att jag med mina erfarenheter och min person skulle göra ett mycket bra jobb i uppdrag inom mjukvara/IT/dataanalys eller liknande.
Jag hoppas att jag har den kompetens och profil ni letar efter och att jag kan få komma på en intervju för att diskutera tjänsten och dess innehåll vidare!


\end{cvletter}

%----------------------------------------------------------------------------------------

\makeletterclosing % Print the signature and enclosures

P.S. Vid eventuell anställning skulle jag vilja bli placerad i Stockholmsområdet där jag bor nu. % i Stocksund. %där jag bor för tillfället.

\end{document}